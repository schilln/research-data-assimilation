\documentclass[12pt]{article}

\newcommand{\commandprependpath}{}
\usepackage{style}

\title{Identifying Models of Fluid Flow with Nudging and Gradient-based Optimization}
\author{Dr. Nathan Schill man}
\date{the future}

\begin{document}
\maketitle

\begin{abstract}
    Data assimilation is the process of using observational data to improve mathematical models, in turn enabling, for example, prediction of future data or identifying the model describing the data.
    One approach, referred to as ``nudging,'' pushes a model's outputs toward observations, and recent research has revealed a way to use gradient-based methods to optimize model parameters.
    My research has focused on implementing this algorithm, and next steps include applying the method to partial differential equations such as those describing fluid flow.
\end{abstract}

Data assimilation is an important topic in many areas of applied mathematics.
The central issue is incorporating data into a model in some way that improves the model's quality.
The AOT algorithm [cite] is one general approach in which incomplete data are used to ``nudge'' a simulated model (in the form of a differential equation) toward the observable portion of nature.
We hope that if the model resembles nature enough and the observations are complete enough, then the simulation will converge to nature.

However, models generally don't resemble nature exactly.
The ``relax-then-punch'' approach [cite] assumes that the model has the same form as nature but differs in the values of some (finite number of) parameters.
Previous research has found positive results in converging to nature [cite].

Yet even so, in many applications an artificial model is likely to not even have the same form as nature, so that there are no ``true'' parameters to recover or converge to.
This paper looks at the performance of the ``relax-then-punch'' approach in this scenario.
The sections are organized as follows:
\begin{itemize}
    \item In section 1 we present results of numerical simulations, showing the convergence of parameters to values that minimize the difference between nature and the simulated system.
    \item In section 2 we present rigorous justification for the approach along with bounds on the error between nature and the simulated system as a function of the parameters and properties of the system.
\end{itemize}

\end{document}
